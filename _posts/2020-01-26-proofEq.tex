\documentclass[10pt,a4paper,oneside]{article}
\usepackage[utf8]{inputenc}
\usepackage[portuguese]{babel}
\usepackage[T1]{fontenc}
\usepackage{amsmath}
\usepackage{mathtools}
\usepackage{amsfonts}
\usepackage{amssymb}
\usepackage{graphicx}
\usepackage{lmodern}
\author{marcos}

\DeclareMathOperator{\taninv}{tan^{-1}}

\begin{document}
\section{Demonstrações e cálculos}
Da figura tal ao observamos o modelo ptolomaico, podemos traçar segmentos de retas estudar os ângulos e formar triângulos que nos fornecerão as medidas que desejarmos dos movimentos planetários no epiciclo e deferente, levando em conta os parâmetros do modelo.

\subsection{Equação dois}
Com o auxílio da figura tal, podemos demonstrar a equação 2, ou seja, basta analisar o $\Delta Tc'P$. Antes devemos verificar o $\Delta Cc'P$ e obter os lados $\overline{Pc'}$ e $\overline{Cc'}$, ou seja,

\begin{equation}
\overline{Pc'}=r\sin\alpha \quad\mbox{e}\quad \overline{Cc'}=r\cos\alpha
\label{02a}
\end{equation}

Agora é possível voltar ao triângulo retângulo $\Delta Tc'P$ para obter a longitude de $P$  a partir da Terra ($T$) devida ao ângulo $\beta$, assim,

\begin{equation}
\tan\beta=\frac{\overline{Pc'}}{\overline{Tc'}} \quad\mbox{onde}\quad \overline{Tc'}=\overline{Cc'}+\overline{TC}
\label{02b}
\end{equation}

Agora substituindo \eqref{02a} em \eqref{02b} considerando que $\overline{TC}=R$, ou seja, o raio do deferente\footnote{No caso não é considerada a excentricidade e a Terra é o centro do deferente}, obtemos a equação 02,

\begin{equation}
\tan\beta=\frac{r\sin\alpha}{r\cos\alpha+R} \implies \beta=\taninv\frac{r\sin\alpha}{r\cos\alpha+R} \quad\blacksquare
\label{02c}
\end{equation}

\subsection{Equação quatro}
Essa equação fornece a relação entre a longitude do centro do epiciclo e a terra, considerando os excêntricos e o ponto equante, bem como os ângulos formados. A partir do ponto $q$ e analisando $\Delta TqC$, é possível observar o teorema seguinte,

\begin{equation}
\overline{TC}^{2}=\overline{Tq}^{2}+\overline{qC}^{2}
\label{04a}
\end{equation}

Considerando a congruência de ângulos \textit{opv}, podemos associar o ângulo $\theta$ ao $\Delta TqE$ e obter a seguinte relação trigonométrica,

\begin{equation}
\sin\theta=\frac{\overline{Tq}}{2e} \implies \overline{Tq}=2e\sin\theta
\label{04b}
\end{equation}

Para o lado $\overline{qC}$ devemos considerar que $\overline{qC}=\overline{Cq'}+\overline{qq'}$. Primeiro, partindo do $\Delta Dq'C$, é possível fazermos a seguinte análise,

\begin{equation}
\begin{aligned}
R^{2}&=\overline{Cq'}^{2}+\overline{Dq'}^{2} \quad\mbox{onde}\quad \overline{Dq'}=e\sin\theta\\
\overline{Cq'}&=(R^{2}-e^{2}\sin^{2}\theta)^{1/2}
\end{aligned}
\label{04c}
\end{equation}

Para completar a equação \eqref{04a} resta o seguimento $\overline{qq'}$, que por semelhança de triângulo podemos conjecturar que $\overline{qq'}=\overline{Eq'}=e\cos\theta$, por fim, temos,

\begin{equation}
\overline{TC}^{2}=4e^{2}\sin^{2}\theta+\lbrace(R^{2}-e^{2}\sin^{2}\theta)^{1/2}+e^{2}\cos^{2}\theta\rbrace^{2} \quad\blacksquare
\label{04d}
\end{equation}


\end{document}